\documentclass[multip]{SGGW-thesis}
\title{Implementacja sieciowej gry wideo z wykorzystaniem silnika Unreal Engine 4}
\author{Maciej Wygoda}
\date{2017}
\university{Szkoła Główna Gospodarstwa Wiejskiego\\w Warszawie}
\dep{Wydział Zastosowań Informatyki i Matematyki}
\Etitle{Implementation of an online video game using Unreal Engine 4}
\album{172407}
\thesis{Praca dyplomowa inżynierska}
\course{Informatyka}
\promotor{dr. Bartłomieja Kubicy}
\pworkplace{Wydział Zastosowań Informatyki i Matematyki\\Katedra Zastosowań Informatyki}

\usepackage{enumitem}
%\usepackage{hyperref}
\PassOptionsToPackage{hyphens}{url}\usepackage{hyperref} %https://tex.stackexchange.com/a/3034

\begin{document}
\maketitle
\twoppage{Maciej Wygoda}{172407}{ktore rodzialy + strony}{Marcin Szadkowski}{wpisz swoj numer albumu}{ktore rodzialy + strony}
\statementpage
\abstractpage
{Implementacja sieciowej gry wideo z wykorzystaniem silnika Unreal Engine 4}
{Niniejsza praca jest opisem implementacji sieciowej gry wideo z wykorzystaniem silnika Unreal Engine 4. Zawiera opis silnika, procesu projektowania i implementowania gry, prezentuje jej architekturę oraz zastosowane rozwiązania.}
{Unreal Engine 4, tworzenie gier wideo, gamedev, gra wideo}
{Implementation of an online video game using Unreal Engine 4}
{This study is a description of an implementation of an online video game using Unreal Engine 4. It describes the engine, the processes of designing and implementing the game and also presents the game's architecture and applied solutions.}
{Unreal Engine 4, game development, gamedev, video game}

\tableofcontents

\chapter{Wstęp}
Gry wideo stanowią rozrywkę dla coraz szerszego grona odbiorców, a sama branża nieustannie rośnie, o czym najlepiej świadczy fakt, iż pod względem wygenerowanych przychodów prześcignęła już branże filmową oraz muzyczną~\cite{nasdaq-video-games-industry}. Gry coraz częściej postrzegane są jako nowoczesne medium przekazu oraz forma wyrazu artystycznego i poruszają tematy dotychczas zarezerwowane dla literatury i kinematografii. W związku z tym istnieje wiele narzędzi wspierających tworzenie gier, a jednym z najpopularniejszych jest silnik {\em Unreal Engine 4} (zwany dalej "UE4").
\section{Cel i zakres pracy}
Motywację do stworzenia niniejszej pracy stanowiły dla nas chęć zgłębienia technologii UE4, podjęcia technicznego wyzwania, jakie stawia zaprogramowanie gry wideo oraz pasja do gier. Na całą pracę składa się zaprojektowanie i zaimplementowanie gry z użyciem UE4 oraz podstawowy opis silnika i implementacji gry. Ponadto może ona służyć za przykład i drogowskaz dla osób chcących napisać własną grę.

\chapter{Unreal Engine 4}
\section{Czym jest silnik gry?}
Przez pojęcie ,,silnik gry`` rozumie się zbiór funkcji i narzędzi ({\em ang. framework}) wspierający tworzenie gier. Musi on oferować przede wszystkim renderowanie grafiki, dźwięku i obsługę sterowania aczkolwiek obecnie najpopularniejsze silniki posiadają znacznie więcej funkcji, a są to między innymi obsługa sieci, symulacja fizyki, edytory shaderów i efektów cząsteczkowych, produkcja przerywników filmowych oraz obsługa wielu platform na przykład komputerów, konsol czy urządzeń mobilnych takich jak smartfony. Każdy popularny silnik dystrybuowany jest wraz z edytorem będacym graficznym interfejsem między programistą, a funkcjami silnika.
\newline Wykorzystanie jednego silnika do stworzenia wielu różnych gier znacząco skraca okres produkcji i stanowi powszechną w branży praktykę.\cite{learning-unreal}\cite{wiki-game-engine}

\section{Komponenty UE4}
tu chodzi o ficzery unreala, info m.in. z naszej prezentacji na seminarium oraz
\newline\url{https://www.unrealengine.com/en-US/features}
\newline\url{https://docs.unrealengine.com/latest/INT/Engine/index.html}

\section{Konwencja}
\url{https://docs.unrealengine.com/latest/INT/Gameplay/Framework/index.html}
\newline\url{https://docs.unrealengine.com/latest/INT/Gameplay/Framework/QuickReference/index.html}
\newline obowiazkowo obrazek z dolu strony :D

\chapter{thesis\_1, rozdzial dot. naszej gry}

\chapter{Kolejny rozdział}

\begin{thebibliography}{9}

\bibitem{learning-unreal}
Joanna Lee, \textit{Learning Unreal Engine Game Development}, Packt Publishing, 2016

\bibitem{wiki-game-engine}
\textit{,,Game engine``, Wikipedia}, \url{https://en.wikipedia.org/wiki/Game_engine} {\mbox(dostęp 29.12.2017)}

\bibitem{nasdaq-video-games-industry}
Trevir Nath, \textit{Investing in Video Games: This Industry Pulls In More Revenue Than Movies, Music},
\url{http://www.nasdaq.com/article/investing-in-video-games-this-industry-pulls-in-more-revenue-than-movies-music-cm634585} {\mbox(dostęp 29.12.2017)}

\end{thebibliography}

\beforelastpage

\end{document} 