\documentclass[multip]{SGGW-thesis}
\title{Implementacja sieciowej gry wideo z wykorzystaniem silnika Unreal Engine 4}
\author{Maciej Wygoda}
\date{2017}
\university{Szkoła Główna Gospodarstwa Wiejskiego\\w Warszawie}
\dep{Wydział Zastosowań Informatyki i Matematyki}
\Etitle{Implementation of an online video game using Unreal Engine 4}
\album{172407}
\thesis{Praca dyplomowa inżynierska}
\course{Informatyka}
\promotor{dr. Bartłomieja Kubicy}
\pworkplace{Wydział Zastosowań Informatyki i Matematyki\\Katedra Zastosowań Informatyki}

\usepackage{enumitem}
%\usepackage{hyperref}
\PassOptionsToPackage{hyphens}{url}\usepackage{hyperref} %https://tex.stackexchange.com/a/3034

\begin{document}
\maketitle
\twoppage{Maciej Wygoda}{172407}{ktore rodzialy + strony}{Marcin Szadkowski}{wpisz swoj numer albumu}{ktore rodzialy + strony}
\statementpage
\abstractpage
{Implementacja sieciowej gry wideo z wykorzystaniem silnika Unreal Engine 4}
{Niniejsza praca jest opisem implementacji sieciowej gry wideo z wykorzystaniem silnika Unreal Engine 4. Zawiera opis silnika, procesu projektowania i implementowania gry, prezentuje jej architekturę oraz zastosowane rozwiązania.}
{Unreal Engine 4, tworzenie gier wideo, gamedev, gra wideo}
{Implementation of an online video game using Unreal Engine 4}
{This study is a description of an implementation of an online video game using Unreal Engine 4. It describes the engine, the processes of designing and implementing the game and also presents the game's architecture and applied solutions.}
{Unreal Engine 4, game development, gamedev, video game}

\tableofcontents

\chapter{Wstęp}
Gry wideo stanowią rozrywkę dla coraz szerszego grona odbiorców, a sama branża nieustannie rośnie, o czym najlepiej świadczy fakt, iż pod względem wygenerowanych przychodów prześcignęła już branże filmową oraz muzyczną~\cite{nasdaq-video-games-industry}. Gry coraz częściej postrzegane są jako nowoczesne medium przekazu oraz forma wyrazu artystycznego i poruszają tematy dotychczas zarezerwowane dla literatury i kinematografii. 
\newline Tworzenie gier wideo ({\em ang. game development}) to obszerne zagadnienie łączące w sobie wiele dziedzin. Od strony technicznej są to między innymi grafika komputerowa, inżynieria oprogramowania, programowanie komputerów, bezpieczeństwo komputerowe, matematyka. W związku z tym, że stworzenie gry to proces długi i skomplikowany, istnieje wiele narzędzi wspierających go, a jednym z najpopularniejszych jest silnik {\em Unreal Engine 4} (zwany dalej "UE4").
\section{Cel i zakres pracy}
Głównym celem niniejszej pracy jest rozwój wiedzy o procesie tworzenia gier wideo. Ponadto motywację stanowiły chęć zgłębienia technologii UE4, podjęcia technicznego wyzwania, jakie stawia zaprogramowanie gry wideo oraz pasja do gier. Na całą pracę składa się zaprojektowanie i zaimplementowanie gry z użyciem UE4 oraz podstawowy opis silnika i implementacji gry. 
\newline Uwagę skupiono między innymi na poznawaniu działania i efektywnym wykorzystywaniu technologii UE4 oraz oprogramowania do modelowania i animacji Blender oraz zdobyciu doświadczenia w pracy zespołowej.
% oraz opracowaniu metodyki pracy, która pozwoli na wielokrotne wykorzystanie napisanego kodu oraz skrócenie czasu tworzenia oprogramowania.
\newline Praca ta może z powodzeniem służyć za przykład i drogowskaz dla osób chcących napisać własną grę.

\chapter{Unreal Engine 4}
\section{Czym jest silnik gry?}
Przez pojęcie ,,silnik gry`` rozumie się zbiór funkcji i narzędzi ({\em ang. framework}) wspierający tworzenie gier. Musi on oferować przede wszystkim renderowanie grafiki, dźwięku i obsługę sterowania aczkolwiek obecnie najpopularniejsze silniki posiadają znacznie więcej funkcji, a są to między innymi obsługa sieci, symulacja fizyki, edytory shaderów i efektów cząsteczkowych, produkcja przerywników filmowych oraz obsługa wielu platform na przykład komputerów, konsol czy urządzeń mobilnych takich jak smartfony. Każdy popularny silnik dystrybuowany jest wraz z edytorem będacym graficznym interfejsem między programistą, a funkcjami silnika.
\newline Wykorzystanie jednego silnika do stworzenia wielu różnych gier znacząco skraca okres produkcji i stanowi powszechną w branży praktykę.\cite{learning-unreal}\cite{wiki-game-engine}

\section{Funkcje silnika Unreal Engine 4}
%tu chodzi o ficzery unreala, info m.in. z naszej prezentacji na seminarium oraz
%\newline\url{https://www.unrealengine.com/en-US/features}
%\newline\url{https://docs.unrealengine.com/latest/INT/Engine/index.html}
UE4 swoją popularność zawdzięcza między innymi otwartemu źródłu, co w pewnym stopniu umożliwia producentom gier dostosowanie silnika do własnych potrzeb na przykład poprzez programowanie narzędzi dla mniej technicznych członków zespołu czy modyfikacje w działaniu silnika. Ponadto UE4 jest w stanie renderować grafikę bardzo zbliżoną do fotorealizmu, co w połączeniu z szeroką gamą oferowanych funkcji sprawia, że poza grami korzysta się z niego na przykład do produkcji spotów i aplikacji reklamowych. \cite{the-human-race}\cite{ikea-vr}
\newline
\newline Oprócz podstawowych funkcji takich jak renderowanie grafiki czy obsługa sterowania do dyspozycji oddane zostały między innymi \cite{ue4-features}:
\begin{description}
\item[Symulacja fizyki:]UE4 korzysta z silnika fizyki {\em PhysX 3.3} dzięki czemu wiarygodnie symuluje kolizje obiektów i inne oddziaływania fizyczne. Producenci mają również możliwość modyfikowania panujących zasad celem lepszego przedstawienia własnej wizji.
\item[Edytor interfejsu użytkownika:] Interfejs stanowi istotny element w komunikacji między grą, a grającym. UE4 zapewnia rozbudowany edytor pozwalający na tworzenie między innymi takich elementów interfejsu jak {\em HUD (head-up display)} czy menu.
\item[Drzewa behawioralne:]Stanowią one podstawę sztucznej inteligencji w UE4 i pozwalają na zaprogramowanie zachowania postaci sterowanych przez komputer w zależności od odbieranych przez nie bodźców i stanu sceny.
\clearpage \item[Sequencer:]Jest to narzędzie służące do produkcji przerywników filmowych. Jego obsługa przypomina pracę z oprogramowaniem do montażu filmów i modelowania 3D. Edytor ten uwzględnia elementy takie jak oś czasu, ujęcia kamery, szkielety i animacje obiektów.
\item[Networking:]https://docs.unrealengine.com/latest/INT/Gameplay/Networking/index.html
\item[Analiza wydajności:]https://docs.unrealengine.com/latest/INT/Engine/Performance/index.html
\item[Edytor materiałów:]https://docs.unrealengine.com/latest/INT/Engine/Rendering/Materials/index.html
\item[Blueprinty:]
\end{description}

\section{Konwencja}
\url{https://docs.unrealengine.com/latest/INT/Gameplay/Framework/index.html}
\newline\url{https://docs.unrealengine.com/latest/INT/Gameplay/Framework/QuickReference/index.html}
\newline obowiazkowo obrazek z dolu strony :D

\chapter{thesis\_1, rozdzial dot. naszej gry}

\chapter{Kolejny rozdział}

\begin{thebibliography}{9}

\bibitem{learning-unreal}
Joanna Lee, \textit{Learning Unreal Engine Game Development}, Packt Publishing, 2016

\bibitem{ue4-features}
\textit{Engine Features}, \url{https://docs.unrealengine.com/latest/INT/Engine/index.html} (dostęp 30.12.2017)

\bibitem{wiki-game-engine}
\textit{,,Game engine``, Wikipedia}, \url{https://en.wikipedia.org/wiki/Game_engine} {\mbox(dostęp 29.12.2017)}

\bibitem{nasdaq-video-games-industry}
Trevir Nath, \textit{Investing in Video Games: This Industry Pulls In More Revenue Than Movies, Music},
\url{http://www.nasdaq.com/article/investing-in-video-games-this-industry-pulls-in-more-revenue-than-movies-music-cm634585} {\mbox(dostęp 29.12.2017)}

\bibitem{the-human-race}
\textit{The Human Race – An Inside Look at the Technology Behind the Groundbreaking Real-Time Film from Epic Games, The Mill and Chevrolet}, \url{https://www.unrealengine.com/en-US/showcase/the-human-race-an-inside-look-at-the-technology-behind-the-groundbreaking-real-time-film-from-epic-games-the-mill-and-chevrolet} (dostęp 30.12.2017)

\bibitem{ikea-vr}
\textit{VIRTUAL REALITY - INTO THE MAGIC}, \url{http://www.ikea.com/ms/en_US/this-is-ikea/ikea-highlights/Virtual-reality/index.html} (dostęp 30.12.2017)

\end{thebibliography}

\beforelastpage

\end{document} 