\documentclass[multip]{SGGW-thesis}
\title{Implementacja sieciowej gry wideo z wykorzystaniem silnika Unreal Engine 4}
\author{Maciej Wygoda}
\date{2017}
\university{Szkoła Główna Gospodarstwa Wiejskiego\\w Warszawie}
\dep{Wydział Zastosowań Informatyki i Matematyki}
\Etitle{Implementation of an online video game using Unreal Engine 4}
\album{172407}
\thesis{Praca dyplomowa inżynierska}
\course{Informatyka}
\promotor{dr Bartłomieja Kubicy}
\pworkplace{Wydział Zastosowań Informatyki i Matematyki\\Katedra Zastosowań Informatyki}

\begin{document}
\maketitle
\twoppage{Maciej Wygoda}{172407}{ktore rodzialy + strony}{Marcin Szadkowski}{wpisz swoj numer albumu}{ktore rodzialy + strony}
\statementpage
\abstractpage
{Implementacja sieciowej gry wideo z wykorzystaniem silnika Unreal Engine 4}
{Niniejsza praca jest opisem implementacji sieciowej gry wideo z wykorzystaniem silnika Unreal Engine 4. Zawiera opis silnika, procesu projektowania i implementowania gry, prezentuje jej architekturę oraz zastosowane rozwiązania.}
{Unreal Engine 4, tworzenie gier wideo, gamedev, gra wideo}
{Implementation of an online video game using Unreal Engine 4}
{This study is a description of an implementation of an online video game using Unreal Engine 4. It describes the engine, the processes of designing and implementing the game and also presents the game's architecture and applied solutions.}
{Unreal Engine 4, game development, gamedev, video game}

\tableofcontents

\chapter{Wstęp}
Dlaczego robienie gier jest super i jak gry wideo ratują świat przed zagładą.

\section{Podrozdział}
To jest podrozdział.

\section{Kolejny podrozdział}
To jest kolejny podrozdział. Spis treści tworzy się automatycznie dopóki zaznaczamy gdzie zaczynają się rozdziały i podrozdziały.

\chapter{Jakiś rozdział}

\chapter{Kolejny rozdział}

\end{document} 